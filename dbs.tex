\documentclass[a4paper,12pt]{article}

% Slovenské prostredie
\usepackage[slovak]{babel}
\usepackage[utf8]{inputenc}
\usepackage[T1]{fontenc}

% Okraje
\usepackage[a4paper,margin=2cm]{geometry}

% Lepšie odkazy
\usepackage[hidelinks]{hyperref}

% Rámčeky
\usepackage[most]{tcolorbox}

\tcbset{
	frame code={},
	colback=gray!5,
	colframe=black,
	boxrule=0.6pt,
	arc=2mm,
	outer arc=2mm,
	boxsep=4pt,
	left=6pt,right=6pt,top=6pt,bottom=6pt
}

% Zvýraznenie kódu (vyžaduje kompiláciu s -shell-escape)
\usepackage{minted}

% Krajsie nadpisy (voliteľné)
\usepackage{titlesec}
\titleformat{\section}{\Large\bfseries}{\thesection}{1em}{}

\begin{document}
	
	\section{Úvod}
	
	\begin{tcolorbox}
		programujem v Pythone
		
		stlač ENTER
	\end{tcolorbox}
	
	Po stlačení klávesu ENTER sa okno zatvorí.
	
	\begin{tcolorbox}[title=print()]
		\begin{itemize}
			\item vypisuje hodnoty výrazov, ktoré sú uvedené medzi zátvorkami
			\item hodnoty sa vypíšu oddelené medzerami
		\end{itemize}
	\end{tcolorbox}
	
	\begin{tcolorbox}[title=input()]
		\begin{itemize}
			\item funkcia načíta vstup od používateľa a vráti reťazec
		\end{itemize}
	\end{tcolorbox}
	
	Funkciu \texttt{input()} môžeme otestovať aj v priamom režime:
	
	\noindent
	\begin{minted}[fontsize=\small]{python}
		>>> input()
		píšem nejaký text
		'píšem nejaký text'
	\end{minted}
	
	\section{Typy údajov}
	
	Typ môžeme zistiť pomocou funkcie \texttt{type()}:
	
	\begin{minted}[fontsize=\small]{python}
		>>> type(42)
		<class 'int'>
	\end{minted}
	
\end{document}
