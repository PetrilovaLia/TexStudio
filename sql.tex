\documentclass[a4paper,12pt]{article}

% Slovenské prostredie
\usepackage[slovak]{babel}
\usepackage[utf8]{inputenc}
\usepackage[T1]{fontenc}

% Okraje
\usepackage[a4paper,margin=2cm]{geometry}

% Lepšie odkazy
\usepackage[hidelinks]{hyperref}

% Rámčeky
\usepackage[most]{tcolorbox}

% Zvýraznenie kódu (vyžaduje kompiláciu s -shell-escape)
\usepackage{minted}

% Krajsie nadpisy (voliteľné)
\usepackage{titlesec}
\titleformat{\section}{\Large\bfseries}{\thesection}{1em}{}

\begin{document}
	
	\section{História databázových technológií}
	
	Prvé počítačové spracovanie údajov sa datuje do 50. rokov. Za ten čas vzniklo niekoľko databázových modelov, každý z nich ovplyvnil ďalší vývoj.
	
	Základným problémom bolo množstvo dát, ktoré bolo nutne fyzicky niekam uložiť, prístup k takémuto veľkému množstvu dát a časová náročnosť. \\
	
	Základom \textbf{hierarchického modelu} je stromová štruktúra, v ktoré sú prepojené v klasickom hierarchickom vzťahu rodič-dieťa. Medzi hlavné obmedzenia tohto modelu patrí v prípade zmeny nutnosť prepracovať celé štruktúry databáz. Vzhľadom na jednosmernosť vzťahu rodič-dieťa (väzba typu 1:n) je problematická realizácia komplikovanejších väzieb (typu m:n). Najznámejšou implementáciou bol systém IMS v rámci kozmického projektu Apollo, kde bolo potrebné organizovanie dvoch miliónov súčiastok. Tento databázový model je možné použiť ešte aj dnes, ako napríklad organizačné či skladové zásoby.
	
	Podstatou \textbf{sieťového modelu} je využitie smerov
	
	
	
	
\end{document}
