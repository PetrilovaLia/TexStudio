\documentclass[a4paper,11pt]{report}

% --- Slovenské nastavenia ---
\usepackage[slovak]{babel}
\usepackage[utf8]{inputenc}
\usepackage[T1]{fontenc}
\usepackage[a4paper,margin=2cm]{geometry}

% --- Kapitoly ---
\usepackage{titlesec}
\titleformat{\chapter}[display]
{\normalfont\bfseries}
{\filright\MakeUppercase{KAPITOLA} \thechapter}
{20pt}
{\titlerule\vspace{1ex}\filright}
[\vspace{1ex}\titlerule]

% --- Rámčeky ---
\usepackage[most]{tcolorbox}
\tcbuselibrary{breakable}


% --- Python kód v takom istom boxe ---
\usepackage{minted} % potrebuje -shell-escape
\usepackage{float}
\usepackage{subcaption}


\setminted{
	fontsize=\small,
	xleftmargin=0pt,
	xrightmargin=0pt,
	autogobble
}

% --- Dokument ---
\begin{document}
	
	\chapter{Úvod do Pythonu}
	
	\subsection*{Čo je to programovanie?}
	
	Programovaním možno označiť celý proces od zadania nejakej úlohy programátorovi, analýzy úlohy (s pomocou odborníka v danej oblasti, ktorý nemusí byť programátor) až po zostavenie algoritmu (konkrétneho postupu) a jeho realizáciu prostredníctvom ľubovoľného programovacieho jazyka, teda \textbf{vytvorenie takého postupu (programu), ktorý je schopný počítač vykonať, a tým automatizovane a samostatne riešiť predložený problém}, ďalej tiež jeho testovanie, vylepšovanie a aktualizáciu. Úlohou programátora je aj tvorba dokumentácie, údržba, poskytovanie licencie pre ďalšie zariadenia a podobne.
	
	\subsection*{Jazyk Python}
	Python je moderný programovací jazyk:
	\begin{itemize}
		\item jeho autorom je Guido van Rossum (1989),
		\item aplikácie v ňom vyvíjajú firmy ako Google, Netflix, Spotify, Instagram...
		\item učí sa ako úvodný jazyk na mnohých univerzitách,
		\item je \textbf{multiplatformový}, teda beží na rôznych platformách ako je Windows, Linux, Mac a tiež je to \textbf{open source} program.
	\end{itemize} 
	
	Na rozdiel od iných programovacích jazykov, ktoré vyžadujú kompiláciu(C/C++), jazyk Python je interpreter, teda:
	\begin{itemize}
		\item nevytvára spustiteľný kód (.exe),
		\item Python musí byť v počítači nainštalovaný,
		\item interpreter umožňuje interaktívnu prácu s prostredím.
	\end{itemize}
	
	\noindent\fbox{\begin{minipage}{\textwidth}{\textbf{ Ako ho získať}}
			\begin{itemize}
				\item zo stránky \textbf{https://www.python.org/} stiahnete najnovšiu verziu Pythonu
				\item spustíte inštalačný program (napr. \textbf{python-3.13.7.exe})
			\end{itemize}
	\end{minipage}} \\

	Spustenie je možné buď priamo v príkazovom riadku alebo prostredí IDLE(Python GUI), čo je vývojové prostredie. Po zadaní príkazu 'python' sa vytvára tzv. \textbf{prompt}. Teraz vieme využívať príkazy jazyka Python.
	
	\begin{minted}[frame=single,framesep=5pt]{python}
		C:\Users\natali> python
		Python 3.13.7 (tags/v3.13.7:bcee1c3, Aug 14 2025, 14:15:11) [MSC v.1944 64 bit 
		(AMD64)] on win32 Type "help", "copyright", "credits" or "license" for more 
		information.
		>>>
	\end{minted}
	
	\textbf{Ako to funguje}
			\begin{itemize}
				\item Python je interpreter a pracuje v niekoľkých možných režimoch,
				\item teraz sme ho spustili v \textbf{príkazovom režime}: teda očakáva zadávanie príkazov do riadku a každý príkaz vyhodnotí, prípadne vypíše \textbf{chybovú hlášku},
				\item toto interaktívne okno sa nazýva \textbf{shell}.
			\end{itemize}


	Python je v jednoduchosti kalkulačka a primárne tak aj funguje a teda môžeme zadávať matematické výrazy. Začneme s celými číslami a operáciami s nimi.
	\begin{minted}[frame=single,framesep=5pt]{python}
		>>> 123 + 456
		579
		>>> 1 * 2 * 3 * 4 * 5 * 6
		720
	\end{minted}
	
	\noindent\fbox{\begin{minipage}{\textwidth}{\textbf{ Celé čísla - INTEGER}}
			\begin{itemize}
				\item rovnaký význam ako v matematike, využívame desiatkovú sústavu a môžu byť záporné
				\item ich dĺžka je obmedzená len kapacitou pamäte Pythonu.
				
			\end{itemize}
	\end{minipage}} \\
	
	Pracovať vieme rovnako ako v matematike aj s desatinnými číslami. 
	
	\begin{minted}[frame=single,framesep=5pt]{python}
		>>> 22/7
		3.142857142857143
		>>> 0.1+0.2+0.3+0.4
		1.0
		>>> .1+.2+.3+.4
		1.0
		>>> 9999999999*99999999999.0
		9.9999999989e+20
	\end{minted}
	
	Python umožňuje ignorovať číslo pred desatinnou čiarkou a automaticky predpokladá, že je to nula. Všimnite si tretí príklad, ktorý je rovnaký ako tretí, ale čísla nezačínajú nulou, napriek tomu dostávame rovnaký výsledok. V poslednom príklade môžeme v zápise vidieť 99999999999.0, teda desatinné číslo a  Python automaticky vyhodnotí výsledok ako desatinné číslo. \\
	
	\noindent\fbox{\begin{minipage}{\textwidth}{\textbf{ Desatinné čísla - FLOAT}}
			\begin{itemize}
				\item pravidlá zápisu v jazyku python vyžadujú zápis pomocou desatinnej BODKY,
				\item môžu vzniknúť ako výsledok inej matematickej operácie (delenie),
				\item majú obmedzenú presnosť na niekoľko desatinných miest.				
			\end{itemize}
	\end{minipage}} \\
	
	Pre prácu s textami máme iný dátový typ, ktorý nám umožňuje prácu s tzv. 'textami', ktoré nazývame \textbf{znakové reťazce}. V prípade, že ich zapíšeme do príkazového riadka(shellu), Python ich vyhodnotí výpisom ich hodnoty.
	
	\begin{minted}[frame=single,framesep=5pt]{python}
		>>> 'ahoj'
		'ahoj'
		>>> "hello world"
		'hello world'
		>>> 'úvodzovky "v" ret’azci'
		'úvodzovky "v" ret’azci'
		>>> "a tiež apostrofy 'v' ret’azci"
		"a tiež apostrofy 'v' ret’azci"
		>>>
	\end{minted}
	
	\noindent\fbox{\begin{minipage}{\textwidth}{\textbf{ Znakové reťazce - STRING}}
			\begin{itemize}
				\item ich dĺžka je obmedzená len kapacitou pamäte Pythonu,
				\item je možné ich zapisovať do 'apostrofov' alebo "úvodzoviek",
					\begin{itemize}
						\item oba zápisy sú ekvivalentné a je len na vás, ktorý budete používať,
						\item reťazec však musí začínať a končiť tým istým znakom,
					\end{itemize}
				\item môže obsahovať špeciálne znaky, diakritiku			
			\end{itemize}
	\end{minipage}} \\
	
	V prostredí \textbf{Shell} dokážeme zadávať len výrazy a vypísať ich hodnoty. Pre písanie príkazov a tvorbu programov budeme ďalej využívať prostredie Visual Studio Code, ktoré je praktickejšie, prehľadnejšie a využívané v praxi. 
	
	\subsection*{Visual Studio Code}
	
	\noindent\fbox{\begin{minipage}{\textwidth}{\textbf{ Ako ho získať}}
			\begin{itemize}
				\item zo stránky \textbf{https://code.visualstudio.com/} stiahnete najnovšiu verziu podľa operačného systému,
				\item spustíte inštalačný program (napr. \textbf{VSCodeUserSetup-x64-1.104.3.exe})
			\end{itemize}
	\end{minipage}} \\ \\
	
	Po spustení programu budete vidieť úvodnú obrazovku VS Code, v ktorom budeme pracovať. 
	
	\begin{figure}[H]
		\centering
		\includegraphics[width=0.5\linewidth]{VSCode_uvod.png}
		\caption{Spustenie VSC}
		\label{fig:uvod}
	\end{figure}
	
	Potrebujete si vytvoriť priečinok, do ktorého budete ukladať vaše python súbory. Takéto súbory musia mať príponu \textbf{.py}, aby mohli byť spustené, tieto súbory nazývame \textbf{skripty}. 
	
	\begin{figure}[H]
		\centering
		\subfloat[\centering]{{\includegraphics[width=0.45\linewidth]{open_folder.png} }}%
		\qquad
		\subfloat[\centering]{{\includegraphics[width=0.45\linewidth]{folder.png} }}%
		\label{fig:folders}%
	\end{figure}
	
	Na obrázku (a) vidíte, ako v prostredí otvoríte váš priečinok a na obrázku (b) ako to bude po otvorení priečinku vyzerať. Vo vašom prípade tam súbory s príponou .py nebudú a bude súbor prázdny. Červenou je označená ikonka na vytvorenie nového súboru. 
	
	\begin{minted}[frame=single,framesep=5pt]{python}
		# moj prvy program
		print('Začíname programovať v Pythone')
	\end{minted}
	
	Takýto program spustíme pravým kliknutím myši na \textbf{Run Python File In Terminal}. Otvorí sa nám terminál, v ktorom sa program spustí a vidíme nasledovný výpis. Ak by sme potrebovali z nejakého dôvodu program prerušiť, použijeme klávesovú skratku \textbf{Ctrl + C}. 

	\begin{minted}[frame=single,framesep=5pt]{python}
		PS C:\Users\natal\Desktop\PYTHON\II> & C:/Users/natal/AppData/Local/Programs/Python/
		Python313/python.exe c:/Users/natal/Desktop/PYTHON/II/prvy_program.py
		Začíname programovať v Pythone
	\end{minted}
	
	Keď sa bližšie pozrieme na náš prvý kód, všimneme si, že sa vypísal iba text v apostrofoch. Text 'moj prvy program' je totižto komentár. Pri programovaní sa často využívajú, pretože chceme zachovať funkciu, ale aktuálne len vytvárame program a nechceme spúšťať všetko napr. kvôli prehľadnosti výpisov atď. Na to slúži symbol \textbf{\#}, ktorý napíšeme pomocou klávesovej skratky \textbf{AltGr + x }. Symbol mriežky spôsobí, že všetko čo sa nachádza v danom riadku, program nevykoná. AK by sme potrebovali viacriadkový komentár, využijeme klávesovú skratku. 
	
	\chapter{CYKLUS FOR}
	
	Ak potrebujeme n-krát vypísať ten istý text, môžeme to zapísať napr. takto:
	
	\begin{minted}[frame=single,framesep=5pt]{python}
		print('programujem v Pythone')
		print('programujem v Pythone')
		print('programujem v Pythone')
		print('programujem v Pythone')
		print('programujem v Pythone')
	\end{minted}
	
	\noindent Namiesto zdĺhavého výpisu použijeme cyklus: 
	\begin{minted}[frame=single,framesep=5pt]{python}
		pocet_opakovani = [1,2,3,4,5]
		for vypisany_text in pocet_opakovani:
		    print('programujem v Pythone')
	\end{minted}
	
	\noindent Ako tento cyklus pracuje:
	\begin{itemize}
		\item do premennej \texttt{vypisany\_text} sa bude postupne priraďovať nasledovná hodnota zo zoznamu hodnôt
		\item začíname s prvou hodnotou v zozname
		\item pre každú hodnotu zo zoznamu sa vykonajú príkazy, ktoré sú v \textbf{tele cyklu}, t.j. tie príkazy, ktoré sú odsadené
		\item v našom príklade sa päťkrát vypíše rovnaký text, hodnota iteračnej premennej alebo inak \textbf{premenná cyklu} - \texttt{vypisany\_text}, nemá
		na tento výpis žiadny vplyv
		\item všimnite si znak \texttt{‘:’} na konci riadka s \texttt{for} - ten je tu povinne, bez neho by to nefungovalo
	\end{itemize}
	
	\noindent Telo cyklu \texttt{FOR}:
	\begin{itemize}
		\item tvoria ho príkazy, ktoré sa majú n-krát opakovať; definujú sa \textbf{odsadením} príslušných riadkov (min. jedna medzera, ideálne 4 = jeden tabulátor),
		
		\item ak telo cyklu obsahuje viac príkazov, všetky \textbf{musia} byť odsadené o rovnaký počet medzier,
		
		\item telo cyklu \textbf{nesmie} byť prázdne, musí obsahovať aspoň jeden príkaz
		
		\item niekedy sa môže hodiť \textbf{prázdny príkaz} \texttt{pass}, ktorý nerobí nič, len oznámi čitateľovi, že sme na telo cyklu
		nezabudli, ale zatiaľ tam nechceme mať nič
		
	\end{itemize}
	
	\begin{minted}[frame=single,framesep=5pt]{python}
		for i in 1,2,3:
		    pass
	\end{minted}
	
	\noindent Do výpisu tela cyklu sa často pridávajú "prázdne riadky", ktoré nemajú žiadny význam len sprehľadňujú výpis kódu, takto nejako to môže vyzerať: 
	
	\begin{minted}[frame=single,framesep=5pt]{python}
		for vypis in 1, 1, 1:
		    print('programujem v Pythone')
		    print('~~~~~~~~~~~~~~~~~~~~~')
		
		programujem v Pythone
		~~~~~~~~~~~~~~~~~~~~~
		programujem v Pythone
		~~~~~~~~~~~~~~~~~~~~~
		programujem v Pythone
		~~~~~~~~~~~~~~~~~~~~~
	\end{minted}
	
	\noindent Ak by sa druhý riadok cyklu neodsadil, vyzeralo by to takto: 
	
	\begin{minted}[frame=single,framesep=5pt]{python}
		for vypis in 1, 1, 1:
		    print('programujem v Pythone')
		print('~~~~~~~~~~~~~~~~~~~~~')
		
		programujem v Pythone
		programujem v Pythone
		programujem v Pythone
		~~~~~~~~~~~~~~~~~~~~~
	\end{minted}
	
	\noindent V tele nášho cyklu \texttt{for} vieme tak ako doteraz využívať premenné a to dokonca \textbf{premennú cyklu}, ktorú definujem len pre daný cyklus. Túto premennú si predstavte ako dočasnú pamäť, do ktorej si cyklus ukladá hodnoty, s ktorými pracuje a neskôr ich "zahodí"
	
	\begin{minted}[frame=single,framesep=5pt]{python}
		cisla = [1,2,3,4,5]
		for prem_cyklu in cisla:
		    print(prem_cyklu)
	\end{minted}
	
	\noindent A po spustení programu dostávame:
	
	\begin{minted}[frame=single,framesep=5pt]{python}
		1
		2
		3
		4
		5
	\end{minted}
	
	\noindent Pomocou cyklu \texttt{for} môžeme vypočítať mocniny daných čísel ako:
	
	\begin{minted}[frame=single,framesep=5pt]{python}
		cisla = [2, 5, 7, 11, 13]
		for x in cisla:
		    x2 = x**2
		    print('Druhá mocnina', x, 'je', x2)
	\end{minted}
	
	\begin{minted}[frame=single,framesep=5pt]{python}
		Druhá mocnina 2 je 4
		Druhá mocnina 5 je 25
		Druhá mocnina 7 je 49
		Druhá mocnina 11 je 121
		Druhá mocnina 13 je 169
	\end{minted}
	
	\noindent Premenná cyklu môže nadobúdať hodnoty ľubovoľného typu. Pre výpočet odmocniny napríklad desatinné čísla (float):
	
	\begin{minted}[frame=single,framesep=5pt]{python}
		cisla = [121, 23/7, 14, 9.0]
		for x in cisla:
		    x_odm = x**.5
		print('Druhá mocnina', x, 'je', x2)
	\end{minted}
	
	\begin{minted}[frame=single,framesep=5pt]{python}
		Druhá odmocnina z čísla 121 je 11.0
		Druhá odmocnina z čísla 3.2857142857142856 je 1.8126539343499315
		Druhá odmocnina z čísla 14 je 3.7416573867739413
		Druhá odmocnina z čísla 9.0 je 3.0
	\end{minted}
	
	\noindent Zoznam poznáme aj ako znakový reťazec, teda môžeme pracovať aj s jednotlivými znakmi tohto reťazca:
	
	\begin{minted}[frame=single,framesep=5pt]{python}
		for pismeno in 'Python':
		    print(pismeno*10)

	\end{minted}
	
	\begin{minted}[frame=single,framesep=5pt]{python}
		for pismeno in 'P', 'y', 't', 'h', 'o', 'n':
		print(pismeno*10)
	\end{minted}
	
	\noindent Pre obe verzie kódu dostávame rovnaký výsledok:
	
	\begin{minted}[frame=single,framesep=5pt]{python}
		PPPPPPPPPP
		yyyyyyyyyy
		tttttttttt
		hhhhhhhhhh
		oooooooooo
		nnnnnnnnnn
	\end{minted}
	
	\subsection*{CVIČENIA}
	
	Výpis hviezdičiek

\end{document}
